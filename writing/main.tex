\documentclass{article}
\usepackage{amsmath, amsthm, amsfonts, amssymb}

%\usepackage{hyperref}
\usepackage[capitalize,noabbrev]{cleveref} % must be loaded after hyperref
%\usepackage{cleveref}
\usepackage{natbib}

\usepackage[utf8]{inputenc}
%\usepackage[T1]{fontenc}
%\newcommand{\upsidedownbang}{!`}
\newcommand{\upsidedownbang}{\rotatebox[origin=c]{180}{!}}
\usepackage{anyfontsize}
\usepackage{xspace}
\usepackage{url}

% balance the last page
\usepackage{flushend}

% Lists
%\usepackage[inline,shortlabels]{enumitem}

% Figures
\usepackage{wrapfig}

% Proof trees
\usepackage{mathpartir}

% Symbolx
\usepackage{cmll} % Linear logic symbols

% Quantum
\usepackage{braket}
\newcommand{\Hilb}{\mathbb{H}}
\newcommand{\qubit}{\texttt{qubit}}
\newcommand{\new}[1]{\mathsf{new}~#1}
\newcommand{\meas}[1]{\mathsf{meas}~#1}


% LNL
\newcommand{\LOWER}{\mathsf{Lower}}
\newcommand{\LIFT}{\mathsf{Lift}}
\newcommand{\Lower}[1]{\LOWER~#1}
\newcommand{\Lift}[1]{\LIFT~#1}
\renewcommand{\put}[1]{\mathsf{put}~#1}
\newcommand{\suspend}[1]{\mathsf{suspend}~#1}
\newcommand{\force}[1]{\mathsf{force}~#1}

% General PL
\newcommand{\bool}{\texttt{bool}}
\newcommand{\LET}{\textsf{let}}
\newcommand{\IN}{\textsf{in}}
\newcommand{\letin}[3]{\LET \; #1 = #2 \; \IN \; #3}
\newcommand{\letbang}[3]{\LET \; !#1 = #2 \; \IN \; #3}
\newcommand{\TT}{\textsf{true}}
\newcommand{\FF}{\textsf{false}}
\newcommand{\IF}{\textsf{if}}
\newcommand{\THEN}{\textsf{then}}
\newcommand{\ELSE}{\textsf{else}}
\newcommand{\ifthen}[3]{\IF \; #1 \; \THEN \; #2 \; \ELSE \; #3}

\newcommand\doubleplus{+\kern-1.3ex+\kern0.8ex}

% Appendix
\usepackage{appendix}

%   Tikz
\usepackage{tikz} %commutativity diagrams
\usetikzlibrary{%
  arrows,%
  shapes.misc,% wg. rounded rectangle
  shapes.arrows,%
  shapes.callouts,
  chains,%
  matrix,%
  positioning,% wg. " of "
  scopes,%
  decorations.pathmorphing,% /pgf/decoration/random steps | erste Graphik
  decorations.text,
  shadows%
}

%\newcommand{\code}[1]{\lstinline{#1}}
\newcommand{\code}[1]{{\small\texttt{#1}}}


\newcommand{\function}[2]{#1~#2}

% Latin  Abbr
\newcommand{\etal}{\emph{et al.}\xspace}
\newcommand{\eg}{\emph{e.g.,}\xspace}
\newcommand{\ie}{\emph{i.e.,}\xspace}
\newcommand{\etc}{\emph{etc.}\xspace}

% Categories
\newcommand{\Hask}{\textsc{hask}}
\newcommand{\Lin}{\textsc{linear}}


% Table formatting
\newcommand{\ltwo}[1]{\multicolumn{2}{l}{#1}}
\newcommand{\ctwo}[1]{\multicolumn{2}{c}{#1}}
\usepackage{booktabs}

% Code
\newcommand{\Just}[1]{\texttt{Just}~#1}
\newcommand{\Nothing}{\texttt{Nothing}}


% New theorems
\newtheorem{definition}{Definition}
\newtheorem{proposition}{Proposition}
\newtheorem{lemma}{Lemma}
\newtheorem{note}{Note}




%% listings for Haskell
\usepackage{listings}
\usepackage{MnSymbol}
\lstloadlanguages{Haskell}
\newcommand{\hcode}[1]{\lstinline[basicstyle=\normalsize\ttfamily,keepspaces=true,literate={+}{{$+$}}1 {/}{{$/$}}1 {*}{{$*$}}1
               {=}{{$=$}}1 %{/=}{{$\neg$}}1
               {>}{{$>$}}1 {<}{{$<$}}1 
               {++}{{$+\!\!\!+$}}1 {::}{{$:\!\!\!\!:$}}1 {∷}{{$:\!\!\!\!:$}}1
               {\\\\}{{\char`\\\char`\\}}1
               {->}{{$\rightarrow$}}1 {>=}{{$\geq$}}2 
               {<-}{{$\leftarrow$}}1 {←}{{$\leftarrow$}}1
               {→}{{$\rightarrow$}}1
               {<=}{{$\leq$}}2 {=>}{{$\Rightarrow$}}2 {⇒}{{$\Rightarrow$}}1
               {-<}{{$\leftY$}}1 
               {\ .\ }{{$\circ$}}2 {(.)}{{($\circ$)}}2
               {=>>=}{{$\lbind$}}2
               {<<}{{$\ll$}}2 {>>}{{$\gg$}}2 {>>=}{{$>\!\!>\!\!=$}}2
               {<<<}{{$\lll$}}2 {>>>}{{$\ggg$}}2 {^<<}{{$\hat{}\!\!\ll$}}2 {^>>}{{$\hat{}\!\!\gg$}}2
               {|}{{$\mid$}}1
               {〉}{{$\rangle$}}1 {〈}{{$\langle$}}1
               {undefined}{{$\bot$}}1
               {not\ }{{$\neg$}}1
               {fun\ }{{\textbackslash}}1
               {^}{{\ensuremath{^\wedge}}}1
               {`elem`}{{$\in$}}1
               {⊢}{{$\vdash$}}2
               {forall}{{$\forall$}}1
               {∀}{{$\forall$}}1
               {>!}{{$>!$}}1
               {$$}{{$ \$ $}}1
               {⋯}{{$\cdots$}}2
               {⊸}{{$\multimap$}}1
               {↝}{{$\leadsto$}}1
               {⊗}{{$\otimes$}}1
               {⊕}{{$\oplus$}}1
               {⋓}{{$\Cup$}}1
                {α}{{$\alpha$}}1
                {β}{{$\beta$}}1
                {γ}{{$\gamma$}}1
                {δ}{{$\delta$}}1
                {ε}{{$\epsilon$}}1
                {λ}{{$\lambda$}}1
                {σ}{{$\sigma$}}1
                {ρ}{{$\rho$}}1
                {Γ}{{$\Gamma$}}1
                {⟦}{{$\llbracket$}}1
                {⟧}{{$\rrbracket$}}1
                {τ}{{$\tau$}}1
                {₁}{{$_1$}}1
                {₂}{{$_2$}}1
]{#1}}
\lstnewenvironment{haskell}
    {\lstset{}%
      \csname lst@SetFirstLabel\endcsname}
    {\csname lst@SaveFirstLabel\endcsname}
    \lstset{
      basicstyle=\small\ttfamily,
      escapeinside={/+}{+/},
%      flexiblecolumns=false,
      belowskip={0.8\medskipamount},
      basewidth={0.5em,0.45em},
      keepspaces=true,
      literate={+}{{$+$}}1 {/}{{$/$}}1 {*}{{$*$}}1
               {=}{{$=$}}1 %{≠}{{$\neq$}}1
               {>}{{$>$}}1 {<}{{$<$}}1 
%               {\\}{{$\lambda$}}1
               {++}{{$+\!\!\!+$}}1 {::}{{$:\!\!\!\!:$}}2 {∷}{{$:\!\!\!\!:$}}1
               {\\\\}{{\char`\\\char`\\}}1
               {->}{{$\rightarrow$}}2 {>=}{{$\geq$}}2 
               {<-}{{$\leftarrow$}}2 {←}{{$\leftarrow$}}1
               {→}{{$\rightarrow$}}1
               {<=}{{$\leq$}}2 {=>}{{$\Rightarrow$}}2 {⇒}{{$\Rightarrow$}}1
               {-<}{{$\leftY$}}1 
%               {\ .\ }{{$\circ$}}2 {(.)}{{($\circ$)}}2
               {=>>=}{{$\lbind$}}2
               {<<}{{$\ll$}}2 {>>}{{$\gg$}}2 {>>=}{{$>\!\!>\!\!=$}}2
               {<<<}{{$\lll$}}2 {>>>}{{$\ggg$}}2 {^<<}{{$\hat{}\!\!\ll$}}2 {^>>}{{$\hat{}\!\!\gg$}}2
               {|}{{$\mid$}}1
               {〉}{{$\rangle$}}1 {〈}{{$\langle$}}1
               {undefined}{{$\bot$}}1
               {not\ }{{$\neg$}}1
               {fun\ }{{\textbackslash}}1
               {^}{{$^\wedge$}}1
               {*^}{{$^\wedge$}}2
               {`elem`}{{$\in$}}1
               {⊢}{{$\vdash$}}2
               {forall}{{$\forall$}}1
               {∀}{{$\forall$}}1
               {!}{{$ \$ $}}1
               {*!}{{$ \$ $}}2
               {>!}{{$>!$}}2
               {⋯}{{$\cdots$}}2
               {⊸}{{$\multimap$}}1
               {↝}{{$\leadsto$}}1
               {⊗}{{$\otimes$}}1
               {*⊗}{{$\otimes$}}2
               {*=}{{$=$}}2
               {⊕}{{$\oplus$}}1
               {⋓}{{$\Cup$}}1
               {⊗⊗}{{$\prod$}}2
                {α}{{$\alpha$}}1
                {β}{{$\beta$}}1
                {η}{{$\eta$}}1
                {γ}{{$\gamma$}}1
                {δ}{{$\delta$}}1
                {ε}{{$\epsilon$}}1
                {λ}{{$\lambda$}}1
                {σ}{{$\sigma$}}1
                {τ}{{$\tau$}}1
                {ρ}{{$\rho$}}1
                {Γ}{{$\Gamma$}}1
                {⟦}{{$\llbracket$}}1
                {⟧}{{$\rrbracket$}}1
                {₁}{{$_1$}}1
                {₂}{{$_2$}}1
              }

\definecolor{identifierColor}{rgb}{0.65,0.16,0.16}
\newcommand{\symb}[1]{\textcolor{black}{\ensuremath{#1}}}

\lstnewenvironment{haskellcolor}
    {\lstset{}%
      \csname lst@SetFirstLabel\endcsname}
    {\csname lst@SaveFirstLabel\endcsname}
    \lstset{
      language=Haskell,
      basicstyle=\ttfamily,
      commentstyle=\sf,
      identifierstyle=\color{identifierColor},
      keywordstyle=\color{blue},
      stringstyle=\color{green}\ttfamily,
      deletekeywords={exp,drop,Handle,openFile,writeFile,fail},
%     flexiblecolumns=false,
      escapeinside={/+}{+/},
      belowskip={0.8\medskipamount},
      basewidth={0.5em,0.45em},
      keepspaces=true,
      upquote=true,
      escapechar=\#,
      literate= {+}{{\symb{+}}}1 
                {/}{{\symb{/}}}1 
                {*}{{\symb{*}}}1
                {=}{{\symb{=}}}1 %{≠}{{$\neq$}}1
                {>}{{\symb{>}}}1 
                {<}{{\symb{<}}}1 
%               {\\}{{$\lambda$}}1
                {++}{{\symb{+\!\!\!+}}}1 
                {::}{{\symb{:\!\!\!\!:}}}2 
                {∷}{{\symb{:\!\!\!\!:}}}1
                {\\\\}{{\char`\\\char`\\}}1
                {->}{{\symb{\rightarrow}}}2 
                {>=}{{\symb{\geq}}}2 
                {<-}{{\symb{\leftarrow}}}2 
                {←}{{\symb{\leftarrow}}}1
                {→}{{\symb{\rightarrow}}}1
                {<=}{{\symb{\leq}}}2 
                {=>}{{\symb{\Rightarrow}}}2 
                {⇒}{{\symb{\Rightarrow}}}1
                {-<}{{\symb{\leftY}}}1 
%               {\ .\ }{{$\circ$}}2 {(.)}{{($\circ$)}}2
                {=>>=}{{\symb{\lbind}}}2
                {<<}{{\symb{\ll}}}2 {>>}{{\symb{\gg}}}2 {>>=}{{\symb{>\!\!>\!\!=}}}2
               {<<<}{{\symb{\lll}}}2 {>>>}{{\symb{\ggg}}}2 {^<<}{{\symb{\hat{}\!\!\ll}}}2 {^>>}{{\symb{\hat{}\!\!\gg}}}2
               {|}{{\symb{\mid}}}1
               {〉}{{\symb{\rangle}}}1 {〈}{{\symb{\langle}}}1
               {undefined}{{\symb{\bot}}}1
               {not\ }{{\symb{\neg}}}1
               {fun\ }{{\textbackslash}}1
               {^}{{\symb{^\wedge}}}1
               {*^}{{\symb{^\wedge}}}2
               {`elem`}{{\symb{\in}}}1
               {⊢}{{\symb{\vdash}}}2
               {forall}{{\symb{\forall}}}1
               {∀}{{\symb{\forall}}}1
               {¡}{{\symb{ \$ }}}1
               {*!}{{\symb{ \$ }}}2
               {>!}{{\symb{>!}}}2
               {⋯}{{$\cdots$}}2
               {⊸}{{\symb{\multimap}}}1
               {↝}{{\symb{\leadsto}}}1
               {⊗}{{\symb{\otimes}}}1
               {*⊗}{{\symb{\otimes}}}2
               {*=}{{\symb{=}}}2
               {⊕}{{\symb{\oplus}}}1
               {⋓}{{\symb{\Cup}}}1
               {⊗⊗}{{\symb{\prod}}}2
               {∅}{{$\emptyset$}}1
                {α}{{$\alpha$}}1
                {β}{{$\beta$}}1
                {η}{{$\eta$}}1
                {γ}{{$\gamma$}}1
                {δ}{{$\delta$}}1
                {ε}{{$\epsilon$}}1
                {λ}{{$\lambda$}}1
                {σ}{{$\sigma$}}1
                {τ}{{$\tau$}}1
                {ρ}{{$\rho$}}1
                {Γ}{{$\Gamma$}}1
                {⟦}{{$\llbracket$}}1
                {⟧}{{$\rrbracket$}}1
                {₁}{{$_1$}}1
                {₂}{{$_2$}}1
              }

%% Unicode
\usepackage{newunicodechar}
\let\Alpha=A
\let\Beta=B
\let\Epsilon=E
\let\Zeta=Z
\let\Eta=H
\let\Iota=I
\let\Kappa=K
\let\Mu=M
\let\Nu=N
\let\Omicron=O
\let\omicron=o
\let\Rho=P
\let\Tau=T
\let\Chi=X

\newunicodechar{Α}{\ensuremath{\Alpha}}
\newunicodechar{α}{\ensuremath{\alpha}}
\newunicodechar{Β}{\ensuremath{\Beta}}
\newunicodechar{β}{\ensuremath{\beta}}
\newunicodechar{Γ}{\ensuremath{\Gamma}}
\newunicodechar{γ}{\ensuremath{\gamma}}
\newunicodechar{Δ}{\ensuremath{\Delta}}
\newunicodechar{δ}{\ensuremath{\delta}}
\newunicodechar{Ε}{\ensuremath{\Epsilon}}
\newunicodechar{ε}{\ensuremath{\epsilon}}
\newunicodechar{ϵ}{\ensuremath{\varepsilon}}
\newunicodechar{Ζ}{\ensuremath{\Zeta}}
\newunicodechar{ζ}{\ensuremath{\zeta}}
\newunicodechar{Η}{\ensuremath{\Eta}}
\newunicodechar{η}{\ensuremath{\eta}}
\newunicodechar{Θ}{\ensuremath{\Theta}}
\newunicodechar{θ}{\ensuremath{\theta}}
\newunicodechar{ϑ}{\ensuremath{\vartheta}}
\newunicodechar{Ι}{\ensuremath{\Iota}}
\newunicodechar{ι}{\ensuremath{\iota}}
\newunicodechar{Κ}{\ensuremath{\Kappa}}
\newunicodechar{κ}{\ensuremath{\kappa}}
\newunicodechar{Λ}{\ensuremath{\Lambda}}
\newunicodechar{λ}{\ensuremath{\lambda}}
\newunicodechar{Μ}{\ensuremath{\Mu}}
\newunicodechar{μ}{\ensuremath{\mu}}
\newunicodechar{Ν}{\ensuremath{\Nu}}
\newunicodechar{ν}{\ensuremath{\nu}}
\newunicodechar{Ξ}{\ensuremath{\Xi}}
\newunicodechar{ξ}{\ensuremath{\xi}}
\newunicodechar{Ο}{\ensuremath{\Omicron}}
\newunicodechar{ο}{\ensuremath{\omicron}}
\newunicodechar{Π}{\ensuremath{\Pi}}
\newunicodechar{π}{\ensuremath{\pi}}
\newunicodechar{ϖ}{\ensuremath{\varpi}}
\newunicodechar{Ρ}{\ensuremath{\Rho}}
\newunicodechar{ρ}{\ensuremath{\rho}}
\newunicodechar{ϱ}{\ensuremath{\varrho}}
\newunicodechar{Σ}{\ensuremath{\Sigma}}
\newunicodechar{σ}{\ensuremath{\sigma}}
\newunicodechar{ς}{\ensuremath{\varsigma}}
\newunicodechar{Τ}{\ensuremath{\Tau}}
\newunicodechar{τ}{\ensuremath{\tau}}
\newunicodechar{Υ}{\ensuremath{\Upsilon}}
\newunicodechar{υ}{\ensuremath{\upsilon}}
\newunicodechar{Φ}{\ensuremath{\Phi}}
\newunicodechar{φ}{\ensuremath{\phi}}
\newunicodechar{ϕ}{\ensuremath{\varphi}}
\newunicodechar{Χ}{\ensuremath{\Chi}}
\newunicodechar{χ}{\ensuremath{\chi}}
\newunicodechar{Ψ}{\ensuremath{\Psi}}
\newunicodechar{ψ}{\ensuremath{\psi}}
\newunicodechar{Ω}{\ensuremath{\Omega}}
\newunicodechar{ω}{\ensuremath{\omega}}
%
\newunicodechar{𝒰}{\ensuremath{\mathcal{U}}}
\newunicodechar{ℕ}{\ensuremath{\mathbb{N}}}
%
\newunicodechar{¬}{\ensuremath{\neg}}
\newunicodechar{•}{\ensuremath{\bullet}}
\newunicodechar{≈}{\ensuremath{\approx}}
\newunicodechar{≅}{\ensuremath{\cong}}
\newunicodechar{≡}{\ensuremath{\equiv}}
\newunicodechar{≤}{\ensuremath{\le}}
\newunicodechar{≥}{\ensuremath{\ge}}
\newunicodechar{≠}{\ensuremath{\neq}}
\newunicodechar{∀}{\ensuremath{\forall}}
\newunicodechar{∃}{\ensuremath{\exists}}
\newunicodechar{±}{\ensuremath{\pm}}
\newunicodechar{∓}{\ensuremath{\pm}}
\newunicodechar{·}{\ensuremath{\cdot}}
\newunicodechar{⋯}{\ensuremath{\cdots}}
\newunicodechar{…}{\ensuremath{\ldots}}
\newunicodechar{∷}{\ensuremath{~\mathrel{:\!\!\!:}~}}
\newunicodechar{×}{\ensuremath{\times}}
\newunicodechar{∞}{\ensuremath{\infty}}
\newunicodechar{→}{\ensuremath{\to}}
\newunicodechar{←}{\ensuremath{\leftarrow}}
\newunicodechar{⇒}{\ensuremath{\Rightarrow}}
\newunicodechar{↦}{\ensuremath{\mapsto}}
\newunicodechar{↝}{\ensuremath{\leadsto}}
\newunicodechar{∨}{\ensuremath{\vee}}
\newunicodechar{∧}{\ensuremath{\wedge}}
\newunicodechar{⊢}{\ensuremath{\vdash}}
\newunicodechar{⊣}{\ensuremath{\dashv}}
\newunicodechar{∣}{\ensuremath{\mid}}
\newunicodechar{∈}{\ensuremath{\in}}
\newunicodechar{⊆}{\ensuremath{\subseteq}}
\newunicodechar{⋓}{\ensuremath{\Cup}}
\newunicodechar{∉}{\ensuremath{\not\in}}

\newunicodechar{⊸}{\ensuremath{\multimap}}
\newunicodechar{⊗}{\ensuremath{\otimes}}
\newunicodechar{⊕}{\ensuremath{\oplus}}
\newunicodechar{〈}{\ensuremath{\langle}}
\newunicodechar{⟨}{\ensuremath{\langle}}
\newunicodechar{⟩}{\ensuremath{\rangle}}
\newunicodechar{〉}{\ensuremath{\rangle}}
\newunicodechar{¡}{\ensuremath{\upsidedownbang}}
\newunicodechar{∘}{\ensuremath{\circ}}
\newunicodechar{†}{\ensuremath{\dagger}}
\newunicodechar{⊤}{\ensuremath{\top}}
\newunicodechar{⊥}{\ensuremath{\bot}}
\newunicodechar{≠}{\ensuremath{\neq}}
\newunicodechar{¡}{\ensuremath{!}}
\newunicodechar{⧺}{\ensuremath{\doubleplus}}

\newunicodechar{₁}{\ensuremath{_1}}

\usepackage{stmaryrd}
\newunicodechar{⟦}{\ensuremath{\llbracket}}
\newunicodechar{⟧}{\ensuremath{\rrbracket}}

%% COMMENTS 
% \usepackage{etoolbox} % replaces ifthen package
% \newtoggle{comments}
% \toggletrue{comments}

% \iftoggle{comments}{
%   \newcommand{\jennifer}[1]{\textbf{\textcolor{red}{[ #1 --- Jennifer]}}}
%   \newcommand{\steve}[1]{\textbf{\textcolor{blue}{[ #1 --- Steve]}}}

%   \newcommand{\fixme}[1]{\textbf{\textcolor{red}{[ Fixme: #1]}}}
%   \newcommand{\note}[1]{\textbf{\textcolor{green}{[ Note: #1 ]}}}
%   \newcommand{\todo}[1]{\textbf{\textcolor{green}{[ TODO: #1 ]}}}
% }{
%   \newcommand{\jennifer}[1]{}
%   \newcommand{\steve}[1]{}

%   \newcommand{\fixme}[1]{}
%   \newcommand{\note}[1]{}
% }
%% END COMMENTS

%\usepackage{comment}
\usepackage{draft}
%\draftfalse
\drafttrue

\newnote[JP]{jennifer}{red}
\newnote[SZ]{steve}{blue}
\newnote{fixme}{green}
%\newnote{note}{green}
\newnote{todo}{green}
\newnote[Cite:]{tocite}{green}

\makeatletter
% Other comments
\NewStdDefinerWrappers{DraftFormat}{%
 \ifdraft
   \expandafter\@firstoftwo
 \else
   \expandafter\@secondoftwo
 \fi
 {#5}{\@firstofone}%
}

\usepackage[normalem]{ulem}
\usepackage{xcolor}
% Re-implemented from the ``ul'' package, via Yiannis Lazarides's answer to
% Aku's TeX Stack Exchange question ``How to color just the wave produced by the
% ulem package'', available at <http://tex.stackexchange.com/a/7322/1425>.
\newcommand*{\coloruwave}[1]{%
 \bgroup
 \markoverwith{\lower3.5\p@\hbox{\sixly\textcolor{#1}{\char58}}}%
 \ULon}
\NewDraftFormat{\awk}{\coloruwave{red}}
\makeatother

%%% Local Variables:
%%% mode: latex
%%% TeX-master: "main"
%%% End:



\title{Quantum HITs}

\begin{document}

\maketitle

\section{The quantum/non-quantum lambda calculus}

The quantum lambda calculus~\citep{selinger2009} is a linear lambda calculus
with the addition of a few quantum primitives that is universal for quantum
computation. In its original formulation the quantum lambda calculus uses the
$!$ exponential with subtyping to deal with duplicable data, but we will instead
use the adjoint structure of linear/non-linear logic. The result consists of a
standard linear/non-linear type system along with a primitive type of qubits and
a number of primitive operations for manipulating quantum data.
\begin{align*}
    α,β &::= \bool ∣ α → β ∣ α × β ∣ \Lift{τ} ∣ ⋯ \\
    σ,τ &::= \qubit ∣ σ ⊗ τ ∣ σ ⊸ τ ∣  \Lower{α} 
\end{align*}
Syntax can similarly be divided between \emph{non-linear terms} and \emph{linear
  expressions}. Terms are a standard lambda calculus with an additional
construct for suspended expressions.
\begin{align*}
    t &::= \TT ∣ \FF ∣ \ifthen{t}{t_1}{t_0} ∣ λ x.t ∣ t~t' ∣ (t,t') ∣ π_1 t ∣ π_2 t ∣ \suspend{e}
\end{align*}
Expressions include the usual constructors for linear pairs and implication. In
addition, $\force{t}$ forces the evaluation of suspended linear expressions,
$\put{t}$ injects a term into the expression language, and $\LET!$ extracts that
term again. We can eliminate non-linear values, such as booleans, inside of
linear expressions. Finally, the quantum lambda calculus provides three
additional syntactic forms: qubit initialization, qubit measurement, and unitary
transformations.
\begin{align*}
    e &::= λ x.e ∣ e~e' ∣ (e,e') ∣ \letin{(x,y)}{e}{e'} ∣ \force t ∣ \put t ∣
        \letin{!x}{e}{e'} ∣ \ifthen{t}{e}{e'} \\
       &∣ \new t ∣ \meas e ∣ U~e
\end{align*}
Here, $U$ ranges over unitary matrices. Every unitary is associated with a type
$σ$, which we denote $U ∈ 𝒰(σ)$. \cref{fig:LNL_types} shows typing
rules for terms and expressions; we omit rules for the standard connectives.

\begin{figure}
\[ \begin{array}{cc}
    \inferrule*
    { Γ;·  ⊢ e : τ }
    { Γ ⊢ \suspend e : \Lift{τ} }
  &
    \inferrule*
    { Γ ⊢ t : \Lift{τ} }
    { Γ;· ⊢ \force t : τ }
  \\ \\
    \inferrule*
    { Γ ⊢ t : α }
    { Γ;· ⊢ \put t : \Lower{α} }
  &
    \inferrule*
    { Γ;Δ_1 ⊢ e : \Lower{α} \\ Γ,x:α;Δ_2 ⊢ e' : τ}
    { Γ;Δ_1,Δ_2 ⊢ \letin{!x}{e}{e'} : τ }
  \\ \\
    \inferrule*
    { Γ ⊢ t : \bool }
    { Γ;· ⊢ \new t : \qubit }
  &
    \inferrule*
    { Γ;Δ ⊢ e : \qubit }
    { Γ;Δ ⊢ \meas e : \Lower \bool }
  \\ \\
    \inferrule*
    { Γ;Δ ⊢ e : σ \\ U ∈ 𝒰(σ) }
    { Γ;Δ ⊢ U~e : σ }
\end{array} \]
\caption{Typing of QNQ connectives}
\label{fig:LNL_types}
\end{figure}

We can give a pure operational semantics for terms (where $\suspend e$ is a
value) and a probabilistic operational semantics for expressions, where states
are represented by quantum closures.

\begin{definition}
    A quantum closure is a tuple $(Q,L,e)$ where 
    \begin{itemize}
      \item $Q$ is an n-dimensional pure quantum state, so a normalized complex
        vector of size $n^2$; and
      \item $L$ is a list of $n$ linear variables, written $\ket{x_1,…,x_n}$;
      \item $e$ is an expression whose free variables all occur in $L$.
    \end{itemize}
\end{definition}

We write $v^e$ for linear values and $v^t$ for term values. Linear values may
contain free (linear) variables, since they will always be evaluated under
closures.
\begin{align*}
    v^t &::= \TT ∣ \FF ∣ λx.t ∣ (v^t_1,v^t_2) ∣ \suspend e \\
    v^e &::= x ∣ λ x.e ∣ (v^e_1,v^e_2) ∣ \put{v^t}
\end{align*}

For the non-quantum expressions, we can give the usual $β$, in addition to the
usual congruence rules. We choose a call-by-value evaluation semantics, but
could have equivalently (?) chosen call by name. I believe the evaluation order
for terms can be independent of the one for expressions. \todo{Check both
  these facts}
\[ \begin{array}{ccc}
    \inferrule*
    { e_1 ↝ e_1' }
    { e_1 e_2 ↝ e_1' e_2 }
   &
    \inferrule*
    { e_2 ↝ e_2' }
    { v_1 e_2 ↝ v_1 e_2' }
   &
    \inferrule*
    {~}
    { (λ x.e) v ↝ e\{v/x\} }
  \\ \\
    \inferrule*
    { e_1 ↝ e_1' }
    { \letin{!x}{e_1}{e_2} ↝ \letin{!x}{e_1'}{e_2} }
  &
    \inferrule*
    { t ↝ t' }
    { \put t ↝ \put{t'} }
  &
    \inferrule*
    {~}
    { \letin{!x}{\put v}{e} ↝ e\{v/x\} }
\end{array} \]
Similarly for pairs.

A value closure is one of the form $(Q,L,v^e)$. The reduction rules for closures
are probabilistic, where $(Q,L,e) ↝_p (Q',L',e')$ indicates that the probability
of the step is $p$. The reduction rules for closures are given in
\cref{fig:closure_reduction}. For the measurement rules, a quantum state $Q$ can
always be uniquely partitioned around a particular qubit as $α Q_0 + β Q_1$. For
more details see \citet{selinger2009}.


\begin{figure}
\[ \begin{array}{c}
    \inferrule*
    {e ↝ e'}
    {(Q,L,e) ↝_1 (Q,L,e')}
  \\ \\
    \inferrule*
    { \text{$Q'$ is the result of applying $U$ to qubits $L[x_1],…,L[x_m]$ in $Q$} }
    { (Q,L,U(x_1,…,x_n)) ↝_1 (Q',L,(x_1,…,x_n)) }
  \\ \\
    \inferrule*
    { y ∉ \vec{xs} }
    { (Q,\vec{xs},\new b) ↝_1 (Q ⊗ \ket{b}, \vec{xs}⧺y, y) }
  \\ \\
    \inferrule*
    { Q_b = \sum_{j} α^b_j \ket{φ^b}\ket{b}\ket{ψ^b}
    \\
      |\ket{φ^b}| = |\vec{xs}|
    \\
      Q' = \sum_{j} α^\FF \ket{φ^\FF}\ket{ψ^\FF}
    }
    {      (α Q_0 + β Q_1,\vec{xs}⧺x⧺\vec{ys},\meas{x}) 
    ↝_{α^2} (Q',\vec{xs}⧺\vec{ys},\put{\FF}) }
  \\ \\
    \inferrule*
    { Q_b = \sum_{j} α^b_j \ket{φ^b}\ket{b}\ket{ψ^b}
    \\
      |\ket{φ^b}| = |\vec{xs}|
    \\
      Q' = \sum_{j} α^\TT \ket{φ^\TT}\ket{ψ^\TT}
    }
    {      (α Q_0 + β Q_1,\vec{xs}⧺x⧺\vec{ys},\meas{x}) 
    ↝_{β^2} (Q',\vec{xs}⧺\vec{ys},\put{\TT}) }
\end{array} \]
\caption{Probabilistic reduction rules for closures}
\label{fig:closure_reduction}
\end{figure}

\subsection{Equational theory}

What does it mean for two expressions to be equivalent under this semantics?
Certainly, this should encompass $β$ and $η$ equivalence. However, we also
expect other equivalences between quantum programs to hold, as well. For
example:
\begin{itemize}
    \item $U_1 (U_2 e) = (U_1 U_2) e$ (where $U_1 U_2$ denotes matrix
      multiplication)
    \item $(U_1 ⊗ U_2) (e_1,e_2) = (U_1 e_1,U_2 e_2)$
    \item $\meas{(\new{t})} = \put{t}$
    \item $\letin{!\_}{\meas{(U e)}}{e'} = \letin{!\_}{e}{e'}$
    \item $X~(\new{t}) = \new{(¬t)}$, where $X$ is the quantum not gate
    \item $\IF_{U_0,U_1}(\new b,e) = (\new b, \ifthen{b}{U_1 e}{U_0 e})$, where
      $\IF_{U_0,U_1}$ is the unitary gate that applies to a pair $(q,e)$ and
      implements $(q,q\ifthen{q}{U_1 e}{U_0 e})$
    \item $[\letin{!z}{\meas e}{\ifthen{z}{U_1 e'}{U_0 e'}}]
          = [\letin{(x,y)}{\IF_{U_0,U_1}(e,e')}{\letin{!_}{\meas x}{y}}]$
\end{itemize}


Equational theories for quantum languages based on the circuit model have been
explored before by \citet{selinger2004}, as well as by \citet{staton2015}, on
whose complete equational theory our system will be based. Staton gives an
algebraic theory with five ``interesting'' axioms (3-8 here) and seven
``administrative'' axioms, including 1 and 2 here. 

\subsection{Equational theory in QNQ}

First, $β$ and $η$ equivalence:
\[\begin{aligned}
    (λ x.e) e' &=_β e\{e'/x\}x \\
    \letin{(x_1,x_2)}{(e_1,e_2)}{e} &=_β e\{e_1/x_1, e_2/x_2\} \\
    \letin{!x}{\put{t}}{e'} &=_β e'\{t/x\} \\
    \force{\suspend e} &=_β e \\
    \ifthen{\TT}{e}{e'} &=_β e \\
    \ifthen{\FF}{e}{e'} &=_β e' 
\end{aligned} \quad \begin{aligned}
    λ x. f x &=_η f \\
    \letin{(x_1,x_2)}{e}{e'\{(x_1,x_2)/z\}} &=_η e'\{e/z\} \\
    \letin{!x}{e}{e'\{\put{x}/z\}} &=_η e'\{e/z\} \\
    \suspend{\force t} &=_η t \\
    \ifthen{t}{e\{\TT/z\}}{e\{\FF/z\}} &=_η e\{t/z\} \\
\end{aligned} \]

Next, we consider the ``interesting'' axioms from \citet{staton2015}. Unitaries
can be thought of as being generated by\footnotemark the single-qubit unitaries,
plus two monoidal structures: the tensor product $⊗$, and quantum alternation,
$⊕$. Alternation can be thought of as a quantum ``if'' statement:
$(U1 ⊕ U2)(q,q')$ can be thought of as $(q,q\ifthen{q}{U1(q')}{U2(q')})$.


\footnotetext{Of course, the unitaries are not \emph{freely} generated by these
  structures, since the results are not unique. However, this collection does
  construct a universal set.}


The axioms come in two groups. The first describes the behavior of unitaries.
\begin{align}
    X (\new b) &= \new{(¬ b)} \\
    (U_1 ⊕ U_2) (\new{b},e) &= (\new{b}, \ifthen{b}{U_1 e}{U_2 e}) \\
    SWAP(e_1,e_2) &= (e_2,e_1)
\end{align}
The next set of axioms describes the behavior of measurement.
\begin{align}
    \meas{(X e)} &= \letin{!x}{\meas e}{\put{(¬ x)}} \\
    \letin{(a,b)}{(U_1 ⊕ U_2)(e,e')}{\letin{!x}{\meas{a}}{e''}} 
    &= \letin{!x}{\meas e}{\letin{b}{(\ifthen{x}{U_1 e'}{U_0 e'})}{e''}} \\
    \meas{\new b} &= \put{b}
\end{align}

\begin{note}
  Staton calls axiom (3) a structural axiom, not interesting, and does not even
  include axiom (1) because in his algebraic theory, all qubits are initialized
  to 0.
\end{note}

The other axioms, including one of Staton's interesting axioms, can be derived
using the homotopy type theoretic structure of unitaries, described in the next section.


\section{Quotients}


Consider the category $𝒰$ of finite-dimensional Hilbert spaces and unitary maps.
That is, a morphism in $𝒰$ between $H$ and $H'$ is an invertible linear map $f$
such that $f^† = f^{-1}$. Since all unitary maps are isomorphisms, $𝒰$ forms a
groupoid. It is also symmetric monoidal, given by the tensor product, and is in fact 
dagger symmetric. Although $\Hilb$, the category of finite-dimensional Hilbert
spaces and linear maps is compact closed, $𝒰$ is not because the unit
$η_A : 1 → A^* ⊗ A$ and counit $ε_A : A ⊗ A → 1$ are not unitary.


The groupoid-quotient or 1-quotient of a type $A$ by a groupoid $G$ is a higher
inductive type generated by: \todo{cite}
\begin{enumerate}
    \item for each $a : A$, a point $\mathtt{point}~a : \mathtt{quotient}_1 A$;
    \item for every morphism $f : G x y$, a 1-cell
      \[\mathtt{cell}~f : \mathtt{point}~x = \mathtt{point}~y;\]
    \item a proof that equality respects the groupoid structure:
        for $f : G x y$ and $g : G y z$, a 2-cell \texttt{cell\_compose} of type
      \[ \mathtt{cell}(g ∘ f) = \mathtt{cell}~f @ \mathtt{cell}~g; \]
    \item a proof that the type $\mathtt{quotient}_1 A$ is a 1-type.
\end{enumerate}

\paragraph{Recursion principle}

The recursion principle states that given a 1-type C along with
\begin{enumerate}
    \item for each \hcode{a : A}, a point \hcode{C_point a : C};
    \item for each \hcode{f : G a b}, a path 
          \hcode{C_cell f : C_point a = C_point b}; and
    \item For each \hcode{f : G x y} and \hcode{g : G y z}, a path
          \hcode{C_compose : C_cell (g o f) = C_cell f @ C_cell g};
\end{enumerate}
there is a function \hcode{quotient1_rec : quotient₁ G -> C} such that
\begin{haskell}
    quotient1_rec (point a) = C_point a
    transport quotient1_rec (cell f)  = C_cell f
\end{haskell}
%    ap₂ quotient1_rec (cell_compose f g) = something
The computation rule for \hcode{cell_compose} is given in the formal
development; it is somewhat complex to write down.

\paragraph{Induction scheme}

The induction principle states that given a predicate 
\hcode{P :: quotient₁ G -> Type}, along with
\begin{enumerate}
    \item for each \hcode{x :: quotient₁ G}, a proof that \hcode{P x} is a
      1-type;
    \item for each \hcode{a : A}, a proof \hcode{P_point a} that \hcode{P} holds
      on \hcode{point a};
    \item for \hcode{f : R a b}, a proof \hcode{P_cell f} that
      \hcode{cell f # P_point a = P_point b}; and
    \item for \hcode{f : R a b} and \hcode{g : R b c}, a proof 
      \hcode{P_compose f g} that
    \begin{haskell}
  P_cell (g o f) = transport2 (cell_compose f g)(P_point x)
                 @ transport_pp
                 @ ap (P_cell f)
                 @ P_cell g 
    \end{haskell}
\end{enumerate}
then there exists a function \hcode{quotient1_ind : forall x, P x} such that
\begin{haskell}
    quotient1_ind (cell f) = P_cell f
    apD quotient1_ind (cell f) = P_cell f
\end{haskell}
%    apD02 quotient1_ind (cell_compose f g) = something
Again, the computation rule for \hcode{cell_compose} is given in the
development. \todo{This should probably be spelled out here.}

\paragraph{Other induction schemes}

From the induction hypothesis we can prove the recursion scheme, as well as a
number of variations. Consider the following:

Given a function \hcode{f :: A -> B -> C}, where $G_A$, $G_B$, and $G_C$ are
groupoids over $A$, $B$, and $C$ respectively, given:
\begin{enumerate}
    \item for each $α : G_A(a_1,a_2)$ and $β : G_B(b_1,b_2)$, a morphism
        \hcode{map_cell α β} in $G_C (f a_1 b_1, f a_2 b_2)$; and
    \item for $α_1 : G_A(a_1,a_2)$, $α_2 : G_A(a_2,a_3)$, $β_1 : G_B (b_1,b_2)$,
        and $β_2 : G_B (b_2,b_3)$, a path \hcode{map_compose} of
        type
      \begin{haskell}
    map_cell (α₂ o α₁) (β₂ o β₁) = map_cell α₂ β₂ o map_cell α₁ β₁
      \end{haskell}
\end{enumerate}

Then there is a function \hcode{quotient1_map2 :: quotient₁ G_A -> quotient₁ G_B
  -> quotient₁ G_C} such that 
\begin{haskell}
    quotient1_map2 (point a) (point b) = point (f a b)
\end{haskell}
and \jennifer{higher computation rules}

\section{Quotients of quantum types}

Consider the type of quantum data types:
\begin{haskell}
data QType where
  Qubit  :: QType
  Tensor :: QType -> QType -> QType
  Lolli  :: QType -> QType -> QType
  Lower  :: Type -> QType
type q1 ⊗ q2 = Tensor q1 q2
type q1 ⊸ q2 = Lolli q1 q2
\end{haskell}

We map every \hcode{QType} to a finite-dimensional Hilbert space corresponding
to the quantum data it carries---this means that all classical types are mapped
to the unit state.
\begin{haskell}
⟦_⟧ :: Qtype -> VecSpace
⟦Qubit⟧   = 1 ⊕ 1
⟦q1 ⊗ q2⟧ = ⟦q1⟧ ⊗ ⟦q2⟧
⟦q1 ⊸ q2⟧ = ⟦q1⟧ ⊗ ⟦q2⟧
⟦Lower τ⟧ = 1
\end{haskell}

Consider the groupoid $\mathcal{U}$ whose objects are \hcode{QType}s, and whose
morphisms $\mathcal{U}(q_1,q_2)$ are unitary transformations $U : ⟦q_1⟧ ⊸ ⟦q_2⟧$,
when $⟦q_1⟧=⟦q_2⟧$. \jennifer{Restrict morphisms to q1 = q2?} Identity,
composition, and invertibility of $\mathcal{U}$ are all inherited from the
structure of the unitaries.

Define the type \hcode{UType} to be the groupoid-quotient of \hcode{QType} by
$\mathcal{U}$.

The $⊗$ and $⊸$ constructions are lifted to \hcode{UType}s using the
\hcode{quotient1_map2} construction detailed above. Because
for $U_1 : \mathcal{U}(q_1,r_1)$ and $U_2 : \mathcal{U}(q_2,r_2)$,
we obtain $U_1 ⊗ U_2 : \mathcal{U}(q_1 ⊗ q_2, r_1 ⊗ r_2)$, we just need to check
the following property:

\begin{lemma}
    Given $U : \mathcal{U}(q_1,q_2)$, $U' : \mathcal{U}(q_2,q_3)$,
    and $V : \mathcal{U}(r_1,r_2)$, $V' : \mathcal{V}(r_2,r_3)$,
    it should be the case that
    $(U' ∘ U) ⊗ (V' ∘ V) = (U' ⊗ V') ∘ (U ⊗ V)$.
\end{lemma}
This is true; see the Matrix development of
QWire\footnote{\url{github.com/jpaykin/qwire}}.

For the sake of convenience, we use \hcode{Qubit} for \hcode{point Qubit}, etc,
when the usage is clear from context.

\subsection{Quantum terms}


We define quantum expressions indexed by a \hcode{UType} as follows:
\begin{haskell}
data QExp :: UType -> Type where
  Abs   :: (QExp q -> QExp r) -> QExp (q ⊸ r)
  Pair  :: QExp q -> QExp r -> QExp (q ⊗ r)
  New   :: Bool -> QExp Qubit
  Put     :: τ -> QExp (Lower τ)
\end{haskell}
%%  Meas  :: QExp Qubit -> QExp (Lower Bool)
We use higher-order abstract syntax to represent variable bindings. (A linearity
check can be defined using a separate judgment, \hcode{Lin :: Ctx -> UType ->
  Type}.) This way we need not encode a separate judgment for $\alpha$, $\beta$,
or $\eta$ equality. For example, we define function application as follows:
\begin{haskell}
app :: QExp (q ⊸ r) -> QExp q -> QExp r
app (Abs f) e = f e
\end{haskell}
so $β$ equality follows judgmentally, and $η$ equality can be proved by
induction on the expression:
\begin{haskell}
lolli_eta :: ∀ (e :: QExp (q ⊸ r)) -> e = Abs (λ x → app e x)
lolli_eta (Abs f) = refl
\end{haskell}
\jennifer{Is there really a point of doing this?}


We can similarly define both $⊗$ and \hcode{Lower} elimination in this way:
\begin{haskell}
(>!) :: QExp (Lower τ) -> (τ -> QExp q) -> QExp q
Put a >! f = f a
\end{haskell}

We can similarly prove $η$ equality, as well as commuting conversions:
\begin{haskell}
lower_eta :: ∀ (e :: QExp (Lower τ)) -> e = LetBang e (λ x → Put x)
lower_eta (Put a) = refl

lower_cc :: ∀ (e :: QExp (Lower σ)) (f :: σ -> QExp (Lower τ)) (f' :: τ -> QExp q)
             (e >! f) >! f' = e >! (λ x → f x >! f')
lower_cc (Put a) = refl -- LHS = f a >! f'
                        -- RHS = f a >! f'
\end{haskell}

Somewhat surprisingly, we also define measurement by case analysis:
\begin{haskell}
meas :: QExp Qubit -> QExp (Lower Bool)
meas (new b) = put b
\end{haskell}

Critically we \emph{do not} include unitary transformations as \hcode{QExp}
constructors, or by case analysis on expressions. Instead, unitary
transformations are encoded in the non-trivial equalities between
\hcode{UTypes}.
\begin{haskell}
unitary :: ∀ {q1 q2} (u : q1 = q2) -> QExp q1 -> QExp q2
unitary refl e = e -- in other words, exactly 'transport'
\end{haskell}

This means that, for example, \hcode{meas (unitary H (New false))} will
\emph{not} reduce to a value, as it shouldn't! 

\jennifer{I don't think the HOAS approach really makes sense, but we can just
  have a simple lambda calculus with deterministic operational semantics.}
\jennifer{Actually, maybe it still will work? see Coq code}

\section{Derived Equations}

\begin{proposition}
  For \hcode{U1 : q1 = q2}, \hcode{U2 : q2 = q3} and \hcode{e : Exp q1},
  \begin{haskell}
    unitary U2 (unitary U1 e) = Unitary (U2 o U1) e
  \end{haskell}
\end{proposition}
\begin{proof} 
    By induction on \hcode{U1}.
\end{proof}

\begin{proposition}
  For \hcode{U1 : q1 = q1'} and \hcode{U2 : q2 = q2'}, 
  \begin{haskell}
    unitary (U1 ⊗ U2) (e1,e2) = (unitary U1 e1, unitary U2 e2)
  \end{haskell}
    where \hcode{U1 ⊗ U2 : q1 ⊗ q2 = q1' ⊗ q2'} is defined by induction on
    \hcode{U1} or \hcode{U2}.
\end{proposition}
\begin{proof}
    By induction on \hcode{U1} and \hcode{U2}
\end{proof}

\begin{proposition}
  For \hcode{U : q1 = q2}, 
  \begin{haskell}
    meas_any (unitary U e) >! (fun _ => e') = meas_any e >! (fun _ => e')
  \end{haskell}
  where \hcode{meas_any : Exp q -> Exp (Lower (toClassical q))}
  extends measurement to all of the qubit values in an expression.
  \jennifer{this could just be the default, for that matter}
\end{proposition}
\begin{proof}
  By induction on \hcode{U}. Notice that this proof would not go through for
  single-qubit measurement, because it is not possible to do induction on
  \hcode{U : Qubit = Qubit}. This makes it hard to extend this approach to
  the quantum alternation lemmas e.g. (5).
\end{proof}


\bibliography{bibliography}
\bibliographystyle{plainnat}
\end{document}


%%% Local Variables:
%%% mode: latex
%%% TeX-master: t
%%% End:
